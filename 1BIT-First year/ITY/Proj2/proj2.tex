\documentclass[twocolumn,11pt,a4paper]{article}

\usepackage{times,amsfonts}
\usepackage[left=1.5cm,text={18cm, 25cm},top=2.5cm]{geometry}
\usepackage[czech]{babel}
\usepackage[utf8]{inputenc}
\usepackage[IL2]{fontenc}
\usepackage{amsthm,amsmath,amssymb}



\theoremstyle{definition}
\newtheorem{thm}{Theorem}[section]
\newtheorem{defn}[thm]{Definice}
\newtheorem{alg}[thm]{Algoritmus}
\newtheorem{exmp}[thm]{Example}
\newtheorem{veta}{Veta}

\begin{document}
\begin{titlepage}

\begin{center}
\Huge\textsc{Fakulta informačních technologií\\
Vysoké učení technické v Brně} \\
\vspace{\stretch{0.382}}
\LARGE Typografie a publikování - 2. projekt \\
Sazba dokumentů a matematických výrazů \\
\vspace{\stretch{0.618}}
\end{center}
{\Large 2014 \hspace{130mm}
Marek Marušic}
\end{titlepage}

\section*{Úvod}
V této úloze si vyzkoušíme sazbu titulní strany, matematických vzorců, prostředí a dalších textových struktur obvyklých pro technicky zaměřené texty (například rovnice (1) nebo definice 1.1 na straně 1).

Na titulní straně je využito sázení nadpisu podle optického středu s využitím zlatého řezu. Tento postup byl probírán na přednášce.

\section{Matematický text}
Nejprve se podíváme na sázení matematických symbolů a výrazů v plynulém textu. Pro množinu $V$ označuje card($V$) kardinalitu $V$.
Pro množinu $V$ reprezentuje $V^*$ volný monoid generovaný množinou $V$ s operací konkatenace.
Prvek identity ve volném monoidu $V^*$ značíme symbolem $\varepsilon$.
Nechť $V^+=V^*-\{\varepsilon\}$. Algebraicky je tedy $V^+$ volná pologrupa generovaná množinou $V$ s operací konkatenace.
Konečnou neprázdnou množinu $V$ nazvěme $abeceda$.
Pro $\omega\in V^*$ označuje $|\omega|$ délku řetězce $\omega$. Pro $W \subseteq V$ označuje occur($\omega,W$) počet výskytů symbolů z $W$ v řetězci $\omega$ a sym($\omega,i$) určuje $i$-tý symbol řetězce $\omega$; například sym($abcd,3$) $=c$.

Nyní zkusíme sazbu definic a vět s využitím balíku\texttt{amsthm}.
\begin{defn}
\emph{Bezkontextová gramatika} je čtveřice $G=(V,T,P,S)$, kde $V$ je totální abeceda,
$T\subseteq V$ je abeceda terminálů, $S\in (V-T)$ je startující symbol a $P$ je konečná množina pravidel
tvaru $q:A\rightarrow \alpha$, kde $A\in (V-T)$, $\alpha\in V^*$ a $q$ je návěští tohoto pravidla. Nechť $N=V-T$ značí abecedu neterminálů.
Pokud $q:A \rightarrow \alpha \in P,\gamma,\delta \in V^*$,$G$ provádí derivační krok z $\gamma A\delta$ do $\gamma\alpha\delta$ podle pravidla $q:A\rightarrow \alpha$, symbolicky píšeme 
$\gamma A \delta\Rightarrow \gamma \alpha\delta [q:A\rightarrow \alpha ]$ nebo zjednodušeně $\gamma A \delta \Rightarrow \gamma \alpha \delta$. Standardním způsobem definujeme $\Rightarrow^m$, kde $m\geq 0$ . Dále definujeme 
tranzitivní uzávěr $\Rightarrow^+$ a tranzitivně-reflexivní uzávěr $\Rightarrow^*$.
\end{defn}

Algoritmus můžeme uvádět podobně jako definice textově, nebo využít pseudokódu vysázeného ve vhodném prostředí (například \texttt{algorithm2e}).


\begin{alg}
\emph{Algoritmus pro ověření bezkontextovosti gramatiky. Mějme gramatiku} $G = (N, T, P, S)$.
\emph{
\begin{enumerate}
\item Pro každé pravidlo $p \in P$ proveď test, zda $p$ na levé straně obsahuje právě jeden symbol z $N$.
\item  Pokud všechna pravidla splňují podmínku z kroku 1, tak je gramatika $G$ bezkontextová.
\end{enumerate}
}

\end{alg}

\begin{defn}
\emph{Jazyk} definovaný gramatikou $G$ definujeme jako $L(G)=\{\omega \in T^*|S\Rightarrow ^* \omega \}$ .
\end{defn}

\subsection{Podsekce obsahující větu}
\begin{defn}
Nechť $L$ je libovolný jazyk. $L$ je \emph{bezkontextový jazyk,} když a jen když $L=L(G)$, kde $G$ je libovolná bezkontextová gramatika.
\end{defn}

\begin{defn}
Množinu $\mathcal{L}_{CF}=\{L|L$je bezkontextovy jazyk\} nazýváme třídou bezkontextových jazyků.

\end{defn}

\begin{veta}
\emph{Nechť $L_{abc}=\{a^n b^n c^n|n \geq 0\}$. Platí, že $L_{abc} \notin \mathcal{L}_{CF}$.}
\end{veta}

\noindent {\emph{Důkaz.} Důkaz se provede pomocí Pumping lemma pro bezkontextové jazyky, kdy ukážeme, že není možné, aby platilo, což bude implikovat pravdivost věty $1$ .

\section{Rovnice a odkazy}
Složitější matematické formulace sázíme mimo plynulý text. Lze umístit několik výrazů na jeden řádek, ale pak je třeba tyto vhodně oddělit, například příkazem \verb|\quad|. 

$$\sqrt[x2]{y^3_0} \quad \mathbb{N} = \{0,1,2,\ldots\}\quad x^{y^y} \neq x^{yy} \quad z_{i_j} \not\equiv z_ij$$

V rovnici ($1$) jsou využity tři typy závorek s různou explicitně definovanou velikostí.

\begin{align}
	\bigg\{\Big[\big(a + b\big) * c\Big]^d + 1\bigg\} =x\\
\lim_{x\rightarrow\infty} \frac{\sin^2x+\cos^x}{4} = y \nonumber
\end{align}

V této větě vidíme, jak vypadá implicitní vysázení limity $\lim_{x\rightarrow\infty} f(n) $ v normálním odstavci textu. Podobně je to i s dalšími symboly jako $\sum{_1^n}$ či $\bigcup{_{A\in \mathcal{B}}}$ . V případě vzorce $\lim\limits_{x \to \infty}\frac{\sin x}{x}$ jsme si vynutili méně úspornou sazbu příkazem \verb|\limits|.

\begin{align}
	\int\limits_a^b f(x)\,\mathrm{d}x = -\int_a^b f(x)\,\mathrm{d}x \\
    \left(\sqrt[5]{x^4}\right)'=\left(x^{\frac{4}{5}}\right)'= \frac{4}{5} x^{-\frac{1}{5}} = \frac{4}{5\sqrt[5]{x}} \\
    \overline{\overline{A \vee B}} = \overline{\overline{A} \wedge \overline{B}}
\end{align}

\section{Matice}
Pro sázení matic se velmi často používá prostředí \texttt{array} a závorky \verb|(\left,\right)|. 

\begin{align}
\left( \begin{array}{ccc@{\ }r}
    a+b & b-a  \\
    \widehat{\xi + \omega} & \hat\pi \\
    \vec a & \overleftrightarrow AC  \\
    0 & \beta  \\
    \end{array} \right) \nonumber \\
    \textbf{A}=
	\begin{array}{||cccc||}
    a_{11} & a_{12} & \ldots & a_{1n}  \\
    a_{21} & a_{22} & \ldots & a_{2n} \\
    \vdots & \vdots & \ddots & \vdots \\
    a_{m1} & a_{m2} & \ldots & a_{mn} \\
    \end{array} \nonumber \\
    \begin{array}{|cc|}
		t & u \\
		v & w \end{array} = tw - uv \nonumber
\end{align}

Prostředí \texttt{array} lze úspěšně využít i jinde.

\begin{align}
    \binom{n}{k} = \left\{ \begin{array}{ll}
    \frac{n!}{k!(n-k)!} & \text{pro } 0 \leq k \leq n\\
    0 & \text{pro } k < 0 \text{ nebo } k > n 
  \end{array} \right.
        \nonumber
\end{align}

\section{Závěrem}
V případě, že budete potřebovat vyjádřit matematickou konstrukci nebo symbol a nebude se Vám dařit jej nalézt v samotném \LaTeX u, doporučuji prostudovat možnosti balíku maker \AmS-\LaTeX.
Analogická poučka platí obecně pro jakoukoli konstrukci v \TeX u.







\end{document}