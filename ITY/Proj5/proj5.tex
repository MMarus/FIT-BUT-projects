\documentclass{beamer}
%
% Choose how your presentation looks.
%
% For more themes, color themes and font themes, see:
% http://deic.uab.es/~iblanes/beamer_gallery/index_by_theme.html
%
\mode<presentation>
{
  \usetheme{default}      % or try Darmstadt, Madrid, Warsaw, ...
  \usecolortheme{default} % or try albatross, beaver, crane, ...
  \usefonttheme{default}  % or try serif, structurebold, ...
  \setbeamertemplate{navigation symbols}{}
  \setbeamertemplate{caption}[numbered]
} 

\usepackage[czech]{babel}
\usepackage[utf8]{inputenc}
\usepackage[IL2]{fontenc}
\usepackage{graphics}
\usepackage{picture}

\providecommand{\uv}[1]{\quotedblbase #1\textquotedblleft}

\title[Your Short Title]{Typografia a~\LaTeX}
\author{Marek Marušic}
\institute{\textsc{Vysoké učení technické v~Brne \\
Fakulta Informačných Technologií}}
\date{\today}

\begin{document}

\begin{frame}
  \titlepage
\end{frame}

% Uncomment these lines for an automatically generated outline.
%\begin{frame}{Outline}
%  \tableofcontents
%\end{frame}

\section{Uvod}

\begin{frame}{Čo nás čaká}

\begin{itemize}
  \item Typografia
  \item Prečo \LaTeX
  \item Ukážka
\end{itemize}

%\vskip 1cm

%\begin{block}{Examples}
%Some examples of commonly used commands and features are included, to help you get started.
%\end{block}

\end{frame}

\section{Typografia}

\begin{frame}{Typografia}

\begin{itemize}
\item Typografia (z~gréckeho \emph{typos} \uv{forma} a~\emph{graphien} \uv{písať })
\item Umelecko-technický odbor, ktorý sa zaoberá: 

\begin{itemize}

    \item prácou s~písmom a~jeho usporiadaním v~grafických prejavoch
    \item problematikou grafickej úpravy tlačených dokumentov
    \item použitím vhodných rezov písma a~usporiadania jednotlivých znakov a~odsekov
    \item dizajnom písma, výberom farebnej schémy dokumentov, ilustrácií, zalamovaním textu do odsekov až po výber papiera pre tlač.
\end{itemize}
  
\item Účelom je, aby bolo možné čítať dokumenty čo najpohodlnejšie a~najprirodzenejšie
\end{itemize}

\end{frame}

\begin{frame}{Typografia pokračovanie}

\begin{itemize}
\item Dnešný typograf na svoju prácu používa zalamovacie programy
\begin{itemize}
\item Adobe InDesign
\item Adobe PageMaker
\item QuarkXPress
\item \TeX
\end{itemize}
\item Na tvorbu písma sa používajú písmové editory
\begin{itemize}
\item FontLab
\item FontCreator
\item FontForge
\end{itemize}


\end{itemize}

\end{frame}

\section{\LaTeX}

\subsection{Prečo \LaTeX}

\begin{frame}{Prečo \LaTeX}
\begin{itemize}
\item Najlepšia voľba pre študentov a~vedcov.
\item Prenositeľný a~univerzálny programovací jazyk na vysadzovanie textu.
\item Možnosť ho editovať v~každom textovom editore. 
\item Hodí sa obzvlášť pre vedecké a~technické dokumenty.
\item Vynikajúca sadzba matematických vzorcov \ref{eq:math}.
\item Vytvára kvalitný výstup.
\item Extrémne stabilný.
\item Spracováva komplexné dokumenty ľahko a~rýchlo bez ohladu na ich veľkosť.
\end{itemize}

\end{frame}

\subsection{Ukážka}

\begin{frame}{Ukážka čo dokáže \LaTeX}
\begin{itemize}
\item Matematické vzorce:
\begin{align}
	\int\limits_a^b f(x)\,\mathrm{d}x = -\int_a^b f(x)\,\mathrm{d}x \\
    \left(\sqrt[5]{x^4}\right)'=\left(x^{\frac{4}{5}}\right)'= \frac{4}{5} x^{-\frac{1}{5}} = \frac{4}{5\sqrt[5]{x}}
\label{eq:math}
\end{align}

\item Obrázky:
\begin{figure}[ht]
\begin{center}
\scalebox{0.1}
{\includegraphics{image.eps}}
\caption{Vektorový obrázok}
\label{fig:Vect}
\end{center}
\end{figure}

\end{itemize}


\end{frame}

\section{Použité zdroje}
\begin{frame}{Použité zdroje}
\begin{itemize}
\item J. Rybička: \emph{ \LaTeX\ pro začátečníky}
\item S.Kottwitz: \emph{\LaTeX\ beginner's guide}
\item Typografia (umenie). \texttt{http://sk.wikipedia.org/wiki/Typografia(umenie)} %\verb|_|
\end{itemize}
\end{frame}

\end{document}
